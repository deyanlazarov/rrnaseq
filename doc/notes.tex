

TODO
------------------
*Empty rows in qc.rds if application of sample_filter between filter settings
*subset (_genes, _samples)
*Add heatmap of rows and cols
*Fix cor.res.list updates such that order of program calls doesnt matter
*Improve as R API (input and return data-structures rather than I/O files)
*Add error-check if sample names differ between meta.rds and rpkm.rds



IDEAS
-------------------
*Make a conf-file and parse the conf and call Rscripts from a python script, here one can also set which programs to call, and possibly in what order.
*class (list): counts, rpkm, meta, qc.meta, ercc



UPDATES
-------------------

v1.0.3
*Bug in updating qc.mat (not resetting the column to zeroes if rerunning and updating qc.mat)

v1.0.2
*Reran test

v1.0.1
*Minor update of README
*Added -f (frac-adjustment) to ercc call in rqc
*Added -e sampledist.cor.res.rds to calls in rqc

v1.0.0
*ercc: filter all data-structures on filtered samples
*read.rpkm: get vector of ERCC ids from all rpkm files instead of only the first file
*gene2nsamples: subset on intersection of meta.mat and rpkm samples
*hclust: increased pdf size
*Updated README (missing getopt package in install instruction)
*Updated .gitignore

v0.1.4
*sampledist.heatmap, sampldist.boxplot, sample.hclust: rm constant columns before correlation calculation, otherwise error is thrown.
*Updated README
*Updated -h info
*rqc: Added several pca calls with different number of pcs to rqc

v0.1.3
*Write qc.tab and not only qc.rds
*pca: Fixed output file-name bug when substituting string 'pca' if two or one PCs.
*sample_heatmap, sample_boxplot, sample_hclust: Added option to save the calculated sample-distance objects (cor.res, col.clust) since for large datasets it takes very long time to calculate.
*mapstats: Work-around for an R bug: Small cex values are silenced using pdf device on OS X. Solution: Increased the pdf size.
*sampledist_boxplot: Changed default of max.cor.cutoff to 0.2.
*get_expr: Add option to set the sample column name in meta.file where sample names will be matched against the rpkmforgenes-file sample names
*get.expr: Add option which reads file-names from a file
*get.meta: Ignore lines with "no data for" (fill = TRUE)
*get.meta: Set sample names for 'no data for' rows output by make_summary_starlog.sh

v0.1.2
*get_expr: stripped '_unique.bam' from sample column names in read.rpkm

v0.1.1
*README updates

v0.1.0
*Tag and "install" the tag on rna (globally)
*Create test dataset which one can run

#Check mapstats for the first files with enough reads and that have ERCC:
#H1007_506_701, H1008_506_702, H1009_506_703

###
#refseq_rpkm.tab
###
cd /mnt/crick/edsgard/nobackup/data/internal/preimplant/hsa/data/rpkmforgenes_star_hg19
cat subdir5.refseq_rpkms.txt | awk -F'\'t '{print NF}' | head #299 samples
cat subdir5.refseq_rpkms.txt | head -4 >head.tmp

cat head.tmp | head -3 | awk -F'\t' -v OFS='\t' '{print $1, $42, $43, $44;}' >head.sel.tab
cat rpkm.tmp | awk -F'\t' -v OFS='\t' '{print $1, $2, $43, $44, $45, $342, $343, $344;}' >rpkm.sel.tab
cat head.sel.tab args.tmp rpkm.sel.tab >test.refseq_rpkms.tab

###
#mapstats log
###
#Log.final.out, qc/geneBody_coverage.geneBodyCoverage.txt, qc/jxn_annotation_ensgene.txt
