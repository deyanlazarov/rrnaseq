

TODO
------------------
*Add several pca calls with different number of pcs to rqc
*Add to README:
-"3 initial steps (prep data)". Example of how to run make_summary_starlog.pl
-mention params
-explain how to qc-filter
-add one or two sentences about what the prg does to each -h
*Add error-check if sample names differ between meta.rds and rpkm.rds



IDEAS
-------------------
*Make a conf-file and parse the conf and call Rscripts from a python script, here one can also set which programs to call, and possibly in what order.
*class (list): counts, rpkm, meta, qc.meta, ercc



UPDATES
-------------------

v0.1.3
*Write qc.tab and not only qc.rds
*pca: Fixed output file-name bug when substituting string 'pca' if two or one PCs.
*sample_heatmap, sample_boxplot, sample_hclust: Added option to save the calculated sample-distance objects (cor.res, col.clust) since for large datasets it takes very long time to calculate.
*mapstats: Work-around for an R bug: Small cex values are silenced using pdf device on OS X. Solution: Increased the pdf size.
*sampledist_boxplot: Changed default of max.cor.cutoff to 0.2.
*get_expr: Add option to set the sample column name in meta.file where sample names will be matched against the rpkmforgenes-file sample names
*get.expr: Add option which reads file-names from a file
*get.meta: Ignore lines with "no data for" (fill = TRUE)
*get.meta: Set sample names for 'no data for' rows output by make_summary_starlog.sh

v0.1.2
*get_expr: stripped '_unique.bam' from sample column names in read.rpkm

v0.1.1
*README updates

v0.1.0
*Tag and "install" the tag on rna (globally)
*Create test dataset which one can run

#Check mapstats for the first files with enough reads and that have ERCC:
#H1007_506_701, H1008_506_702, H1009_506_703

###
#refseq_rpkm.tab
###
cd /mnt/crick/edsgard/nobackup/data/internal/preimplant/hsa/data/rpkmforgenes_star_hg19
cat subdir5.refseq_rpkms.txt | awk -F'\'t '{print NF}' | head #299 samples
cat subdir5.refseq_rpkms.txt | head -4 >head.tmp

cat head.tmp | head -3 | awk -F'\t' -v OFS='\t' '{print $1, $42, $43, $44;}' >head.sel.tab
cat rpkm.tmp | awk -F'\t' -v OFS='\t' '{print $1, $2, $43, $44, $45, $342, $343, $344;}' >rpkm.sel.tab
cat head.sel.tab args.tmp rpkm.sel.tab >test.refseq_rpkms.tab

###
#mapstats log
###
#Log.final.out, qc/geneBody_coverage.geneBodyCoverage.txt, qc/jxn_annotation_ensgene.txt
